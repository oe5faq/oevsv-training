\mode<article>{
    \section*{Frage 2: Leiter Halbleiter Nichtleiter}
    \addcontentsline{toc}{section}{(2) Leiter Halbleiter Nichtleiter}
}

\mode<article>{
        Die Leifähigkeit von Materialien ist von vielen verschiedenen Faktoren wie Licht, Temperatur etc abhängig.
    }
\begin{frame}{Leiter Halbleiter Nichtleiter}
    %\framesubtitle{Frage 2: Leiter Halbleiter Nichtleiter}

\mode<article>{

        \textbf{Leiter} sind Materialien, die den elektrischen Strom sehr gut leiten, z. B. alle Metalle, Kohlen, Säuren. Damit elektischer Strom fließen kann, müssen sogenannte freie Ladungsträger zwischen den Atomen vorhanden sein. Die Leitfähigkeit eines Stoffes oder Stoffgemisches hängt von der Verfügbarkeit dieser beweglichen Ladungsträger ab. Dies können locker gebundene Elektronen wie beispielsweise in Metallen, aber auch Ionen in organischen Molekülen sein. Sehr gute Leiter sind (in der Reihenfolge abnehmender Leitfähigkeit) Silber, Kupfer, Aluminium, Gold und Messing.
        
        \textbf{Halbleiter} sind Materialien, die ihre Leitfähigkeit aufgrund physikalischer (Druck, Temperatur, Licht etc.) oder elektrischer Einflüsse verändern können, z. B. Silizium und Germanium. Siehe Frage 22.
        
        \textbf{Nichtleiter} sind Materialien, die den elektrischen Strom sehr schlecht leiten (Isolatoren). Gute Isolatoren sind Glas, Keramik, Kunststoff, Pertinax, Glasfaser-Harz, Teflon, Gummi und trockenes Holz.
        Die elektrische Leitfähigkeit (Konduktivität) ist eine physikalische Größe, die die Fähigkeit eines Stoffes angibt, elektrischen Strom zu leiten. Das Formelzeichen der elektrischen Leitfähigkeit ist o (Sigma).
    }
    \mode<presentation>{
        \frame{}{
            \begin{itemize}
               \item Leiter sind Materialien, die den elektrischen Strom sehr gut leiten
                \item Halbleiter sind Materialien, die ihre Leitfähigkeit aufgrund physikalischer (Druck, Temperatur, Licht etc.) oder elektrischer Einflüsse verändern können (Silizium, Germanium).
                \item Nichtleiter sind Materialien, die den elektrischen Strom sehr schlecht leiten (Isolatoren). Gute Isolatoren sind Glas, Keramik, Kunststoff, Pertinax, Glasfaser-Harz, Teflon, Gummi und trockenes Holz.
            \end{itemize}
        }
    }
    
\end{frame}

